
\chapter{Escolha do Tema}

O primeiro passo de um PFC é a escolha do \textbf{tema}. Essa escolha é feita por meio de um acordo entre orientados e orientador. Por isso, se o aluno deseja um tema específico, deve procurar um orientador que tenha interesse no mesmo.

Normalmente um tema é escolhido de forma geral e depois detalhado. Por exemplo, o aluno pode desejar fazer um PFC sobre “Aprendizado de Máquina” e, detalhando mais tarde, “Classificar automaticamente críticas de livro como favoráveis ou não”.

O tema de um PFC será bom caso seja um tema que permita aos alunos que o irão desenvolver mostrarem a seus avaliadores que dominam os conhecimentos, as habilidades e competências esperadas dos egressos do curso, no nosso caso, do curso de Bacharelado em Ciência da Computação.

A escolha do tema é extremamente importante, pois é um fator essencial na motivação do aluno para fazer o PFC. Um tema que pouco interesse o aluno acabará se tornando um fardo, enquanto um tema que interesse ao aluno pode se tornar um prazer e até mesmo um projeto continuado após a graduação.

\textbf{A temática do PFC tem que estar relacionada ao curso que o aluno está concluindo, e que o desenvolvimento do PFC deve ser conduzido de forma condizente com os preceitos, métodos, técnicas, boas práticas e questões éticas pertinentes à carreira estudada. }

\section{Se você ``está devendo o PFC''}

Alguns alunos não fazem o PFC em tempo hábil e voltam mais tarde em busca de um orientador e de um tema, porém já estão trabalhando e têm pouco tempo disponível. O segredo então é apresentar como projeto final um projeto que tenha realizado no seu trabalho, ou algo que tem forte relação com o que faz no momento. Tentar ``inventar'' um assunto que será mais uma carga adicional de trabalho a sua vida diária é um erro.