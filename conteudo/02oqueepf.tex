\chapter{O que é o Projeto Final de Curso}

O Projeto Final de Curso, ou Projeto Final de Curso (PFC), é uma \textbf{atividade obrigatória} do curso de Bacharelado em Ciência da Computação (BCC) da UFRJ. Ela é classificada como um Requisito Curricular Suplementar.

Como o nome diz, ele é um projeto, realizado por \textbf{um aluno ou um grupo de até 3 estudantes}, que serve para comprovar, de alguma maneira, que os participantes atingiram, em algum grau aceitável, certas competências esperadas do egresso do curso.

Como é um projeto, o PFC é “um esforço temporário para criar um produto, serviço ou resultado exclusivo”[PMBOK]. Uma das principais características de um projeto é ter início e fim, i.e., um tempo determinado. No BCC o tempo esperado de um PFC é de até 1 ano, e o estudante deve se inscrever em uma cadeira chamada Projeto Final de Curso, MABX02, na turma do seu orientador.

Como detalharemos mais tarde, o PFC inclui duas entregas obrigatórias, formais: um texto seguindo normas acadêmicas, e uma apresentação ao vivo. Ambas as entregas são apresentadas a uma banca para avaliadora, que pode decidir entre aprovar o projeto, aprovar condicionado à execução de alterações ou reprovar (em todos os casos, a banca atribui uma nota ao PFC). Normalmente, estas entregas relatam o trabalho feito durante o projeto.

Existem vários tipos de projetos possíveis no BCC. É possível conhecer vários Projetos Finais do passado pelo sistema Pantheon da UFRJ\footnote{\url{https://pantheon.ufrj.br/}}.

\section{Por que o PFC?}

A principal função de um PFC é a formação dos alunos envolvidos. Por isso, não se espera que haja um avanço do conhecimento, na forma de uma contribuição original, ou o desenvolvimento de um produto em estágio final.

Claro que ter resultados interessantes é um subproduto excelente de um PFC, e também um indício de que o aluno teve capacidade para realizar um bom trabalho. Porém, os resultados em si, científicos ou tecnológicos, não são o que mais importa. Podemos dizer que o PFC é mais sobre o caminho do que sobre o ponto de chegada.

Os alunos escolhem o orientador, de acordo com o tema que desejam trabalhar e suas afinidades pessoais. Não é preciso ter vergonha de pedir para ser orientado, isso é uma das funções dos professores e na verdade, uma das melhores.

\textbf{Não é necessário esperar chegar ao último ano do curso para começar o PFC. }

O ideal é que você converse com os professores com antecedência para que possa escolher a melhor opção de trabalho. E não é necessário ficar inscrito 1 ano em PFC antes de apresentá-lo. Vc pode terminar e apresentar o PFC em 6 meses.

\section{Quem está envolvido no PFC}

O PFC tem um, ou mais, personagens principais: o \textbf{estudante ou o grupo} que o realiza. Todo trabalho é responsabilidade deles ou delas. São eles, e elas, os maiores beneficiados do resultado do trabalho, não só pelo aprendizado que o trabalho vai trazer, mas principalmente, e pragmaticamente, por cumprir uma das últimas tarefas necessárias para obter o seu diploma.

As estudantes são guiadas nesse caminho por um \textbf{orientador}. O orientador deve ser um professor do Departamento de Ciência da Computação, com a possibilidade de ter um co-orientador externo ao departamento. Esse orientador estabelecerá, com as estudantes, o escopo e o prazo do projeto, dentro do limite de 2 semestres consecutivos. Ele é o principal responsável, junto com uma banca, pela aprovação e nota do projeto.

Caso a defesa do PFC não não ocorra dentro do tempo regular de 2 semestres consecutivos, o aluno é reprovado e nova inscrição deverá ser feita no período seguinte.

No final do projeto, outros setores da UFRJ vão estar envolvidos, como a biblioteca, que receberá o PFC, e a seção de ensino, que realizará a burocracia que permitirá que as estudantes colem grau.


