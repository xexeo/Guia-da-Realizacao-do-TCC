

\chapter{A Originalidade do Texto}

Apesar do PFC não precisar ser um trabalho original, isto é, é possível fazer algo que já foi feito por outro, tanto o texto quanto a apresentação têm que ser originais. A cópia do trabalho de outro configura plágio.

Basicamente, plagiar significa apresentar como seu trabalho que foi feito e já publicado por outro. No mundo acadêmico, o plágio é considerado uma desonestidade séria e é punida de várias formas, formais e informais, como a exclusão de um curso, a reprovação de um trabalho ou em uma cadeira e até mesmo, em alguns casos, sendo levada à justiça comum.

Isso significa que todo texto para o qual assumimos a autoria deve ser original, sob o risco de incorrer em plágio. Obviamente, não é possível fazer trabalho científico sem se utilizar de ideias e textos de outros autores como ponto de partida e apoio, logo existem regras claras de como realizar citações, isto é, como descrever o trabalho de outro de forma que fique clara a atribuição de autoria.

No Brasil existe uma norma de citação mantida pela ABNT e muitas universidades mantêm versões próprias, possivelmente inspiradas na ABNT. Nessas normas se descrevem, de forma bastante detalhada, as várias maneiras de se declarar uma citação. Nem sempre, porém, fica claro o que é uma citação.

Existem duas formas de citação: a citação direta e a citação indireta.

Na citação direta copiamos diretamente o texto do autor e, por causa disso, devemos marcar de forma clara que estamos fazendo essa cópia. Segundo a norma ABNT isso é feito pelo uso de aspas, quando a citação tem até 3 linhas, ou usando um parágrafo com recuo de 4 cm da margem esquerda (ABNT, 2001).

Ainda de acordo com a ABNT (2001) :
“citações indiretas  (ou livres) são a reprodução de algumas idéias, sem que haja transcrição das palavras do autor consultado. Apesar de ser livre, deve ser fiel ao sentido do texto original. Não necessita de aspas.”
Nos dois parágrafos acima fizemos uma citação indireta ao descrever a direta e uma citação direta ao descrever uma indireta.
