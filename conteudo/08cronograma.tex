\chapter{Um Cronograma}

É razoável imaginar um cronograma mínimo para a realização do PFC. Os seguintes passos devem ser executados seguindo algum ciclo de vida de projeto, seja ele sequencial ou iterativo.

\begin{enumerate}

\item Definição do Problema, que deve ser acordado por todo o grupo e o orientador

\item Revisão Bibliográfica, que normalmente exige duas partes, textos relativos ao problema e textos relativos a técnica de solução utilizada. A escrita do PFC começa aqui.

\item Proposta de Solução, escolhida na literatura ou proposta pelos alunos ou professor

\item Execução da Solução, a fase normalmente mais longa, muitas vezes incluindo uma implementação

\item Avaliação da Solução, onde cada par problema/solução exige uma técnica apropriada.

\item Escrita Final do PFC

\item Preparação da Apresentação

\end{enumerate}

Esses passos podem variar um pouco com o tipo de Projeto Final que está sendo feito. Uma Revisão Sistemática da literatura, por exemplo, é tipo de projeto final em que o que se executa é a pesquisa bibliográfica, de uma forma específica. Porém, mesmo assim, há a necessidade do aluno entender como ela é feita, então a Revisão Bibliográfica a ser feita é sobre a Revisão Sistemática, e a execução é executar essa revisão sobre algum tema escolhido.
