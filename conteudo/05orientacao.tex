

\chapter{O Processo de Orientação}

Ao longo do projeto, o grupo deve se encontrar regularmente com seu orientador. Uma boa frequência é entre cada 15 ou 30 dias. É muito útil fazer um planejamento e um cronograma das principais atividades do PFC no primeiro mês do processo de orientação, logo após ficar bem definido qual será o tema e o escopo do projeto.

Assim se tem uma visão clara do estágio do desenvolvimento do trabalho e do quanto ainda falta para concluir o projeto.

Tanto o orientador quanto os orientados devem estar de acordo acerca do planejamento e do cronograma, evidenciando que todos tenham o mesmo entendimento e expectativas sobre o andamento do trabalho.

Nesses encontros é normal que o grupo apresente o que fez e que o orientador e o grupo acordem em novas tarefas. A cada encontro o projeto deve ter avançado. Fazer encontros onde o grupo “não fez nada” ou claramente fez tudo no dia anterior à noite é ruim para o processo.

Recomendamos fortemente que cada membro do grupo tenha um caderno de anotações do projeto e use esse caderno o tempo todo. Opções semelhantes é usar Google Docs acessíveis pelo celular ou aplicativos de notas.

Alguns orientadores vão montar um cronograma logo no início, outros vão fazer de uma forma mais similar aos métodos ágeis, criando objetivos por ciclos.
