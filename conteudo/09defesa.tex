
\chapter{Como é a Defesa}

A Defesa de Projeto Final é o ponto de clímax do projeto. Normalmente ela é marcada quando o orientador acredita que o PFC está bom para ser aprovado.

Quando seu trabalho estiver quase pronto, começam as conversas sobre a defesa do mesmo. Para a defesa, os estudantes entregam o texto final para a banca com uma antecedência combinada, normalmente em torno de 15 dias.

No dia da defesa o orientador reservará uma sala e os alunos farão a exposição de seu trabalho, com um tempo em torno de 30 a 45 minutos. O orientador definirá esse tempo.

A apresentação é normalmente feita com o apoio de slides, por exemplo feitos em PowerPoint, Google Slides, LaTeX com beamer, etc. Esses slides devem apresentar o trabalho de forma clara, com foco no que foi feito pelos alunos.

Os alunos podem convidar colegas ou outras pessoas para assistir a apresentação, porém é bom avisar para pessoas de fora, como pais e mães, que serão feitas perguntas que fazem críticas ao trabalho.

Uma proporção razoável é que pelo menos 50\% do tempo seja gasto com o trabalho realmente feito, sendo que o resto do tempo fala da motivação, assuntos revisados, etc.

Após a defesa a banca faz perguntas. Essas perguntas servem para elucidar pontos que não ficaram claros no texto ou na apresentação e discutir a qualidade do trabalho. Além disso, a banca analisa a desenvoltura do grupo nas respostas.

Após as perguntas, os alunos e a platéia saem de sala e os professores decidem a nota a ser dada ao projeto.

No dia da defesa os alunos devem levar cópias adicionais da capa, para que sejam assinadas pelos membros da banca.

Além disso, os alunos devem estar preparados para a eventualidades da internet, do projetor ou do computador não funcionar, além de outros problemas físicos. Para isso é importante levar um arquivo com a apresentação em um pen-drive (ou outro método similar), e, se houver algum exemplo do sistema funcionando, ter gravado um filme ou uma sequência de print screens do mesmo.

Um guia para como fazer os slides pode ser encontrado em: \url{https://github.com/xexeo/MaterialEducacional/blob/main/DicasSlidesAcademicos.pdf}

E não se esqueça da foto com a banca para marcar o evento!